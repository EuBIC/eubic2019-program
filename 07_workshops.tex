\section*{\color{eubicRed} Workshop Abstracts}

\subsection*{Day 1 -- 10. January 2017}

\subsubsection*{\color{eubicRed} Introduction to proteomics data analysis}
{\color{eubicGray}Harald Barsnes and Marc Vaudel}

Mass spectrometry based proteomic experiments generate ever larger datasets and,
as a consequence, complex data interpretation challenges. In this course, the
concepts and methods required to tackle these challenges will be introduced. The
course will focus on protein identification and take the participant from the
handling of the raw data to the statistical analysis of the identification data.
The course will rely exclusively on free and user friendly software, all of
which can be directly applied in your lab upon your return from the Winter
School.


\subsubsection*{\color{eubicRed} ELIXIR and de.NBI Hackathon}
{\color{eubicGray} Magnus Palmblad, Jon Ison, Niall Beard, Julian Uszkoreit, Gerhard Mayer and Veit Schwaemmle}

Do you want to learn about the current European initiatives for computational
infrastructure and data standards in life sciences? We aim to make the
computational proteomics community an integrative part of these recent
developments. The hackathon, kindly sponsored by ELIXIR Denmark and in
cooperation with de.NBI, is the first outreach of ELIXIR - the European
Infrastructure for Biological Information - to the Proteomics Community. Current
ELIXIR efforts for establishing sustainable infrastructures within computational
biology will be presented. We will introduce and discuss integration of
computational proteomics into ELIXIR and initiation of common projects in both
research and training. The main focus is on the bio.tools registry for software
annotation, workflow composition, standards for data formats and data uploads.
The participants will be engaged in the discussions and task forces to deepen
collaborations between ELIXIR, de.NBI and experts in computational proteomics.


\subsubsection*{\color{eubicRed} Developing spectral libraries with Progenesis QI for proteomics}
{\color{eubicGray} Waters}

Learn how to use Progenesis QI for proteomics to Quantify and Identify the
peptides and proteins that are significantly changing in your samples, and how
you'll soon be able to develop your own customised spectral libraries to improve
the speed and specificity of your MSe searches. Bring your own laptop and
download the demo software and tutorial from
\url{http://www.nonlinear.com/progenesis/qi-for-proteomics/download/}


\subsubsection*{\color{eubicRed} Highlights of the latest developments of the Proteome Discoverer framework.}
{\color{eubicGray} Thermo}

An overview of the features in the current version, PD 2.1 will be presented
with emphasis on the improvements in quantification. Furthermore we will
highlight the upcoming  new release, including Label Free Quantification and
Cross-linking.

This will be an interactive presentation with time for questions and feedback.


%%%%%%%%%%%%%%%%%%%%%%%%%%%%%%%%%%%%%%%%%%%%%%%%%%%%%%%%%%%%%%%%%%%%%%%%%%%%%%%%
\subsection*{Day 2 -- 11. January 2017}

\subsubsection*{\color{eubicRed} Structural Interactomics by Cross-linking Mass Spectrometry}
{\color{eubicGray} Fan Liu}

Chemical cross-linking combined with mass spectrometry (XL-MS) has emerged as a
powerful approach to investigate protein conformation as well as protein-protein
interactions. Especially in recent year, this technique has moved rapidly
towards the analysis of very large protein assemblies and heterogeneous mixtures
of protein complexes. This course is designed for scientist who is interested in
using this technique to probe the structure of various proteins/protein
complexes and to discover novel protein-protein interactions. The course will
cover most aspects of the XL-MS workflow, including sample preparation,
cross-link enrichment, MS data acquisition and cross-link data analysis.
Furthermore, attendees will also be provided with practical training on XL-MS
data analysis using standalone XlinkX and XlinkX PD node.


\subsubsection*{\color{eubicRed} Global big data challenge}
{\color{eubicGray} Nuno Bandeira}

This "global big data challenge" workshop will focus on three key aspects of
contributing to the global proteomics knowledge base: i) using advanced
algorithms for discovery and inspection of post-translational modifications and
highly-modified peptides; ii) sharing search results with the community at large
and reviewing results contributed by others; iii) reusing spectral libraries for
peptide identification and detection in both DDA and DIA mass spectrometry data.
The workshop will cover topics of relevance to both experimentalists (e.g., how
to critically inspect search results) and bioinformaticians (e.g., how to share
and compare results from new software tools).


\subsubsection*{\color{eubicRed} IPA workshop}
{\color{eubicGray} Mario Ricketts and Andre Koper}

Ingenuity\textregistered{} Pathway Analysis (IPA\textregistered{}) is a powerful
analysis and search tool that helps researchers to uncover the biological
significance of 'omics data and to answer critical questions related to their
studies. Built on extensive and primarily manually curated scientific content
from QIAGEN's distinctive Knowledge Base, IPA's content-aware and causal
analytics assists with the identification of canonical pathways, phenotypic
effects, networks of interacting molecules and putative upstream drivers that
help users interpret various types of 'omics experiments, including measurements
of differentially expressed or phosphorylated proteins. The integration of
different 'omics data is a strong focus for IPA that allows users to visualize
different molecular data together and to gather evidence for a more
comprehensive interpretation of experimental data with just a few mouse clicks.
IPA has been cited in over 16,000 scientific publications.

This workshop will help the attendees gain an overview of IPA's capabilities and
to experience its graphical user interface and approach to the biological
interpretation of proteomics and phospho-proteomics datasets with or without
accompanying gene expression data. As part of the workshop, we will focus on
creating and interpreting IPA analyses and demonstrating tools in IPA to create,
visualize and analyze causal effects of automated and user-defined molecular
networks. At the end of this 4 hour session, each attendee should be able to
upload and analyze their data in IPA in a comprehensive manner.  In addition,
QIAGEN will provide 14 day IPA trials to every attendee of this IPA workshop at
the EuBIC Winterschool 2017.


\subsubsection*{\color{eubicRed} Proteomics without a genome: leveraging RNA-Seq from non-model organisms}
{\color{eubicGray} David Tabb}

As proteomics spreads to an ever-larger number of applications, some researchers
(particularly in agriculture) are hindered by the lack of a high-quality genome
annotation for their species of interest.  In this workshop, we will examine
methods for preparing a draft proteome sequence database from transcriptomic
data collected through high-throughput sequencing (RNA-Seq). These stages
include the following:
\begin{enumerate}
  \item checking the quality of the sequencing reads
  \item assembling the transcripts de novo
  \item translating the transcripts to amino acids
  \item finding nearest annotated taxonomic neighbors
  \item annotating sequences by reciprocal BLAST and InterPro
\end{enumerate}


%%%%%%%%%%%%%%%%%%%%%%%%%%%%%%%%%%%%%%%%%%%%%%%%%%%%%%%%%%%%%%%%%%%%%%%%%%%%%%%%
\subsection*{Day 3 -- 12. January 2017}


\subsubsection*{\color{eubicRed} MaxQuant/Perseus}
{\color{eubicGray} Jürgen Cox}
Description: TBA


\subsubsection*{\color{eubicRed} Automated processing of quantitative proteomics data with OpenMS}
{\color{eubicGray} Oliver Kohlbacher and Julianus Pfeuffer}

The workshops provides a brief introduction to OpenMS, an open-source software
for computational proteomics and metabolomics. Participants will familiarize
themselves with the underlying concepts and get to know a few key tools of the
OpenMS tool collection. We will construct tailor-made automated data analysis
workflows for database search, label-free quantification, data visualization and
quality control in proteomics.

These workflows will be applied to selected example data sets. Particpants are
encouraged to bring their own data and discuss the analyses required with
instructors from the OpenMS team. All software used will be provided and can be
installed on particpants' own computers.


Speakers: Antonio
\subsubsection*{\color{eubicRed} TBA}
{\color{eubicGray} Antonio Fabregat Mundo}

Description: TBA
