\mysection{7}{Workshop Abstracts}

\noindent\textbf{Day 1 -- 15. January 2018}
\addcontentsline{toc}{subsection}{Day 1}

\subsubsection*{\color{eubicRed} Introduction to computational mass spectrometry using OpenMS}
{\color{eubicGray}Hannes Röst}

We will use the OpenMS library to explore mass spectrometric raw data and data
processing basic concepts in mass spectrometry. We will learn about the
visualization tools in OpenMS, the scripting capabilities using Python and the
internal algorithms and datastructures available. We will also talk about the
community and how you can write your own tools in OpenMS and contribute to the
project.

\subsubsection*{\color{eubicRed} Computational introduction into DIA}
{\color{eubicGray}Maarten Dhaenens and Brian Searle}

\subsubsection*{\color{eubicRed} Label-free quantification: concepts and algorithms}
{\color{eubicGray}David Bouyssié}

\subsubsection*{\color{eubicRed} Quantitative proteomics, statistics, clustering and complexes}
{\color{eubicGray}Veit Schwämmle and David Bouyssié}

Basic guidelines and methods for visual inspection of quantitative proteomics data, to apply statistical tests, clustering of multivariate data and quantitative assessment of the behavior of protein complexes

%%%%%%%%%%%%%%%%%%%%%%%%%%%%%%%%%%%%%%%%%%%%%%%%%%%%%%%%%%%%%%%%%%%%%%%%%%%%%%%%
\vspace{1cm}
\noindent\textbf{Day 2 -- 16. January 2019}
\addcontentsline{toc}{subsection}{Day 2}

\subsubsection*{\color{eubicRed} The essentials before and after spectrum identification: Selecting the appropriate database and inference strategy}
{\color{eubicGray}Martin Eisenacher und Julian Uszkoreit}

In this workshop the participants will learn more about two fundamental pillars
of most proteomics studies: the protein databases and the protein inference. We
will discuss and show in hands on tutorials which databases are suitable for
which analyses and what needs to be considered for the right choice.
Furthermore, workflows for performing protein inference using PIA will be
explained and teached hands-on.

Attendees should bring their own laptop for the workshop.

\subsubsection*{\color{eubicRed} Quality Control and Benchmarking of Label-Free Quantification Workflows with LFQBench}
{\color{eubicGray}Stefan Tenzer}

In this workshop, we will introduce the concept of benchmarking label-free quantification workflows using mixed proteome standards.

Applications of LFQBench in instrument benchmarking, quality control and etablishment and validation of quantification workflows will be discussed.

Participants will install and use the open source R-package LFQBench and learn how to interpret the various outputs generated by LFQBench.

Several test datasets will be provided and analyzed.

Please bring your own laptop. Some working experience with R is helpful, but not mandatory for the workshop.

LFQBench can be obtained here: \url{https://github.com/IFIproteomics/LFQbench}


\subsubsection*{\color{eubicRed} Proteome Discoverer 2.3 Workshop}
{\color{eubicGray}Thermo}

New Features in PD 2.3
\begin{itemize}
  \item Statistics and Quantification roll-up strategies for Precursor and Reported based quantification
  \item Advanced featured and nodes (Cross-linking, Top down)
  \item Node programming
  \item Q \& A
\end{itemize}

\subsubsection*{\color{eubicRed} Discovering the open-source Proline software suite, a new efficient and user friendly solution for label-free quantification}
{\color{eubicGray}ProFI}

DDA Label-free quantification based on precursor ion intensity is a widely used
method for quantifying differentially expressed proteins across different
conditions or samples. An ideal software solution should allow the production
of reliable and comprehensive results, and be flexible enough to allow the
integration of existing tools without compromising ease-of-use. To meet these
objectives we developed the Proline software, a next-generation tool based on a
modular data processing toolbox. This tool constitutes a very interesting
alternative to competing solutions, combining robustness, performance,
modularity and user-friendliness.

This workshop will be a good opportunity to discover the data processing
functionalities of Proline and also the various visualization tools integrated
in the Proline-Zero desktop application. After a short introduction of the main
software features, we will follow several tutorials aiming at providing a
global overview of the tool. During this hands-on session, we will run an the
data analysis of a standard dataset composed of an equimolar mixture of 48
human proteins (UPS1, Sigma) spiked at different concentrations into a yeast
cell lysate background.

%%%%%%%%%%%%%%%%%%%%%%%%%%%%%%%%%%%%%%%%%%%%%%%%%%%%%%%%%%%%%%%%%%%%%%%%%%%%%%%%
\vspace{1cm}
\noindent\textbf{Day 3 -- 17. January 2019}
\addcontentsline{toc}{subsection}{Day 3}

\subsubsection*{\color{eubicRed} Network visualization with Cytoscape and stringApp}
{\color{eubicGray}Lars Juhl Jensen}

The workshop will first provide a quick introduction on the Cytoscape network
analysis and visualization tool as well as the Cytoscape stringApp, which makes
it easy to import networks from STRING into Cytoscape. Afterwards, we will move
on to hands-on exercises, which will teach you how to:
\begin{itemize}
  \item retrieve networks for proteins or small-molecule compounds of interest
  \item retrieve networks for a disease or an arbitrary topics in PubMed
  \item layout and visually style the resulting networks
  \item import external data and map them onto a network
  \item perform enrichment analyses and visualize the results
  \item merge and compare networks
  \item select proteins by attributes
  \item identify functional modules through network clustering
\end{itemize}
If time permits, I will also try to demonstrate how Cytoscape can be used to
address some of the challenges that came up during the debate.

\subsubsection*{\color{eubicRed} SELPHI: using data-driven approaches for analysis of phosphoproteomics datasets}
{\color{eubicGray}Evangelia Petsalaki}

Current phosphoproteomics data analysis pipelines focus mostly on identifying
differentially regulated peptides and mapping them on known pathways. This
limits our insight around pathways that are well studied and annotated. SELPHI
aims to take a data driven approach, to help biologists explore the space less
studied in their datasets.

In this workshop I will explain what the aim of SELPHI is and how it works. I
will also describe how to generate files for use with SELPHI and will perform a
walk through of all the different results that you can acquire using this tool.

\subsubsection*{\color{eubicRed} Advanced data acquisition methods with MaxQuant.Live}
{\color{eubicGray}Florian Meier}

MaxQuant.Live (www.maxquant.live) is a freely available software framework for
real-time monitoring of mass spectrometric data and controlling of the data
acquisition. It enables advanced data acquisition strategies on Q Exactive mass
spectrometers such as BoxCar (Meier et al., Nat. Methods 2018) and EASI-tag
quantification (Virreira Winter et al., Nat. Methods 2018) via a user-friendly
graphical interface. Furthermore, it recognizes thousands of peptide precursors
in real-time by live re-calibration in three dimensions. In this workshop, you
will get familiar with the MaxQuant.Live app store and start generating your
own methods, for example a global targeting method for over 20,000 peptides in
a single run.

\subsubsection*{\color{eubicRed} Validation of peptide identifications}
{\color{eubicGray}Matthias Wilhelm}

\begin{enumerate}
  \item Get predicted spectra from ProteomicsDB
  \item Compare results to [proteogenomics/sORF] data using R
  \item Investigate effects of pre-processing on spectra similarity
\end{enumerate}
