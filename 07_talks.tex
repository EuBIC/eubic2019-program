\mysection{6}{Talk Abstracts}

\noindent\textbf{Day 2 -- 16. January 2019}
\addcontentsline{toc}{subsection}{Day 2}

\subsubsection*{\color{eubicRed} XCorDIA: a new database search engine to detect genetic variants from DIA data}
{\color{eubicGray}Brian Searle}

Single nucleotide polymorphisms and other genomic sequence variants can have
profound impact on susceptibility to disease. Even still, most shotgun
proteomics workflows focus on detecting canonical protein sequences found in
FASTA databases. While proteogenomics methods that combine customized exome
sequencing with mass spectrometry are emerging for data dependent acquisition
(DDA), data independent acquisition (DIA) approaches frequently rely on curated
spectrum libraries that lack sequence variants. Moreover, because most variants
result in small retention time and M/Z shifts, these peptides often co-isolate
and fragment together in wide DIA precursor isolation windows. Variant peptides
produce many of the same fragment ions as canonical peptides and confidently
distinguishing different forms is challenging. Moreover, peptide-centric search
engines can produce undetectable false positives using these shared ions when
searching for low mass PTMs or sequence variants caused by either SNPs,
paralogs, or orthologs. We present XCorDIA, a new database search engine that
detects and statistically validates PTMS and peptide variants in PEFF databases
from DIA data. XCorDIA searches for PTMs and sequence variants by batching
peptides that share fragment ions and confirming the presence of specific
variants using a PTM/variant detection algorithm similar to PTM
site-localization algorithms. We validate XCorDIA using methionine oxidized
peptides. Oxidation shifts precursor mass by only +16 Da, which produces mass
shifts similar to most SNPs (e.g. +14 Da, V  L). Without variant-specific
scoring, we find that based on shared fragment ions approximately 1/3rd of
oxidized peptides are incorrectly detected at the retention time corresponding
to the unmodified form. Finally, we demonstrate how XCorDIA detects sequence
variants from ClinVar using clinical amyloidosis samples.

\subsubsection*{\color{eubicRed} Label-free Quantification of Complex Proteomes using Ion-Mobility-based DIA }
{\color{eubicGray}Stefan Tenzer}

Mass spectrometry-based proteomics greatly benefited from recent improvements in instrument performance and the development of bioinformatics solutions facilitating the high-throughput quantification of proteins in complex biological samples.

Unbiased data-independent acquisition (DIA) strategies have gained increased popularity in the field of quantitative proteomics in the last years, as they provide a complete record of all detectable analytes and facilitates precise label-free quantification of highly complex samples.

The integration of ion mobility separation (IMS) into DIA workflows provides an additional dimension of separation to liquid chromatography-mass spectrometry (LC-MS), and increases the achievable analytical depth of DIA approaches.

From a computational aspect, the increased data complexity provides several opportunities for novel data processing approaches, but is also challenging, as multidimensional solutions for peak picking, alignment and visualization have to be implemented.

\subsubsection*{\color{eubicRed} Bioinformatics for Proteomics - any open questions?}
{\color{eubicGray}Martin Eisenacher}

Proteomics, especially with mass spectrometry has reached many milestones.
Several challenges postulated as being show stoppers have been addressed:
identification with limited false positives, quantification, finding "all"
gene-coded proteins, modifications (plus localization), usable standard
formats. In parallel, instruments and algorithms became more sensitive, more
exact and data more sustainable. But there are still some unexplained
phenomenons, all-day questions to solve, closed doors to open. For example, the
increasing mass accuracy creates new challenges to false-discovery rate
estimation. Or, shared peptides could be used for a better quantification. To
open the box of pandora - all our method development in mass spectrometry for
Proteomics may become obsolete some day.


\vspace{1cm}
\noindent\textbf{Day 3 -- 17. January 2019}
\addcontentsline{toc}{subsection}{Day 3}

\subsubsection*{\color{eubicRed} STRING -- Large-scale integration of data and text}
{\color{eubicGray}Lars Juhl Jensen}

Methodological advances have in recent years given us unprecedented information
on the molecular details of living cells. However, it remains a challenge to
collect all the available data on individual genes and to integrate the highly
heterogeneous evidence available with what is described in the scientific
literature. The STRING database aims to address this by consolidating known and
predicted protein–protein association data for a large number of organisms. In
my presentation, I will give an overview of the STRING database and describe
the general approach we use to unify heterogeneous data, provide comparable
quality scores for all evidence types, and automatically mine associations from
the biomedical literature.

\subsubsection*{\color{eubicRed} Prosit: Proteome-wide prediction of peptide tandem mass spectra by deep learning}
{\color{eubicGray}Mathias Wilhelm}

In mass spectrometry-based proteomics, the identification and quantification of
peptides heavily relies on sequence database searching. However, the lack of
accurate predictive models for fragment ion intensities impairs the realization
of the full potential of this approach. Via ProteomicsDB, reference spectra
from the ProteomeTools project and predictions from Prosit are now available,
allowing their integration into various proteomic workflows. This talk will
showcase some of the many applications in DDA, DIA and PRM workflows.

\subsubsection*{\color{eubicRed} Using phosphoproteomics data to study context-specific signalling}
{\color{eubicGray}Evangelia Petsalaki}

Phosphoproteomics data provide a snapshot of the phosphorylation-based
signaling state of cells.  They can therefore be used to dissect the dynamic
networks active in a cell in a given condition. Several methods infer
context-specific signalling networks from phosphoproteomics data by using as
informative priors either existing protein-protein interaction networks or
networks from pathway databases. These suffer from severe study bias and
therefore data-driven analyses could provide more scope for novel discoveries
and an improved understanding of context-specific cell signalling.

To allow non-bioinformaticians to perform data-driven analyses on
phospho-proteomics datasets I developed SELPHI, which performs automated data
integration and correlation-based network inference. We applied SELPHI to
phospho-proteomics data from B cells in variable inhibitor and stimulation
conditions, and we identified a novel substrate recognition motif for the Fes
kinase. Our follow-up study showed that the motif is recognized by the CSK
kinase, and led to explaining the dual oncogenic and tumor suppressive function
of Fes.

Currently, we have taken advantage of published phosphoproteomics datasets to
generate a data-driven kinase signalling network that can be used as an
informative prior for network inference.  I will also present preliminary
results on a new method that uses phosphoproteomics data to derive
context-specific cell signaling networks.

\subsubsection*{\color{eubicRed} Insights into the multi-functioning proteome}
{\color{eubicGray}Kathryn Lilley}

It is well accepted paradox in biology genome size does not correlate with
organismal complexity. In terms of proteins, it could be argued that in higher
organisms the proteome is simply too small for the complex functions that it is
has to perform [Doolittle, W. F. (2013) Proc. Natl Acad. Sci. USA 110, 5294–5300].

The chemical space that the proteome occupies in higher organisms is vastly
expanded by post translational modification, but the numbers and roles of
differently functioning proteoforms in a cell are currently uncertain.

In this talk I will review methods that shed light on different functional
roles of proteins, from establishing multiple subcellular locations of
proteins, to determining additional nucleic acid binding properties of
metabolic enzymes. I will also discuss difficulties in trying to determine
alternative functional roles for proteins.


\vspace{1cm}
\noindent\textbf{Day 4 -- 18. January 2019}
\addcontentsline{toc}{subsection}{Day 4}

\subsubsection*{\color{eubicRed} Ionbot: a novel, fully data-driven search engine for open modification and mutation searches}
{\color{eubicGray}Sven Degroeve}

Modern shotgun proteomics is entirely dependent on accurate search engine tools
to match observed spectra to the peptide sequences that generated them. Here we
focus on the widely applied approach that is based on a target database that
contains all proteins (peptides) expected to be in the sample under
consideration. To accommodate for the high computational cost of matching tens
of thousands of MS/MS spectra, peptides in the target database are typically
considered to be modified by a few of the most common modifications only.

We present Ionbot, a completely new and highly powerful search engine that
allows for matching MS/MS spectra against extremely large target databases
(allowing thousands of potential protein modifications including mutations). It
achieves high processing speeds by implementing a new data-driven approach for
selecting candidate peptides for a given MS/MS spectrum. Further, a novel PSM
scoring function based on predicted MS/MS spectra is presented as a means to
maintain a high degree of sensitivity (at fixed FDR) when handling very large
target databases. Ionbot will be demonstrated to perform very well in open
modification and mutation searches.

\subsubsection*{\color{eubicRed} Trapped ion mobility spectrometry: a new dimension for mass spectrometry-based proteomics}
{\color{eubicGray}Florian Meier}

The fast scan speed of time-of-flight analyzers allows adding ion mobility
spectrometry as a third dimension of separation. Trapped ion mobility
spectrometry (TIMS) is particularly attractive due to its compact design and
highly efficient ion utilization. We have recently introduced a novel scan mode
termed parallel accumulation – serial fragmentation (PASEF), which synchronizes
the release of peptide ions from the TIMS device with the precursor selection
in the quadrupole (PMID: 26538118). In data-dependent acquisition, PASEF
increases the sequencing by more than 10-fold without any loss in sensitivity
(PMID: 30385480). Transferring the PASEF principle to data-independent
acquisition could, in principle, capture a much larger proportion of the
available ion current as compared with classical DIA, thereby improving
sensitivity and acquisition speed several fold. We further demonstrate that
peptide collisional cross sections can be readily measured at the scale of
100,000s with high precision. We conclude that TIMS in combination with PASEF
is an exciting addition to the technological toolbox in proteomics, with many
unique operating modes and applications still left to be explored.
